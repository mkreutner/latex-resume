%-------------------------------------------------------------------------------
%	SECTION TITLE
%-------------------------------------------------------------------------------
\cvsection{Education}

%-------------------------------------------------------------------------------
%	CONTENT
%-------------------------------------------------------------------------------
\begin{cventries}

%---------------------------------------------------------
  \cventry
    {Doctorat - Automatisation et robotique} % Degree
    {IAEM de Lorraine} % Institution
    {Nancy, France} % Location
    {09/2000 - 09/2004} % Date(s)
    {
      \begin{cvitems} % Description(s) bullet points
        \item {Sujet : Perception multisensorielle pour le positionnement d'un véhicule autonome dédié aux personnes handicapées.}
      \end{cvitems}
    }
  
%---------------------------------------------------------
  \cventry
    {DEA ATNS}
    {INP Lorraine} % Institution
    {Nancy, Lorraine} % Location
    {09/1998 - 09/1999} % Date(s)
    {
      \begin{cvitems} % Description(s) bullet points
        \item {Maîtriser les différentes technologies liées à l'automatisation et à la robotique ;}
        \item {Concevoir, développer et optimiser une application incluant l'automatique et la robotique ;}
        \item {Diriger un projet de manière autonome en intégrant la gestion budgétaire, la gestion des fournisseurs ou des sous-traitants ;}
        \item {Gérer les activités du département ;}
        \item {Communiquer et échanger en anglais technique.}
      \end{cvitems}
    }

%---------------------------------------------------------
  \cventry
    {Licence et maîtrise en génie des systèmes industriels, option informatique} % Degree
    {Université de Lorraine} % Institution
    {Metz, France} % Location
    {09/1996 - 09/1998} % Date(s)
    {
      \begin{cvitems} % Description(s) bullet points
        \item {Former des cadres de niveau ingénieur aux compétences en gestion des opérations, en système industriel, en production et en livraison logistique ;}
        \item {avec une forte ouverture aux défis de l’exploitation d’une entreprise ;} 
        \item {avec des compétences transversales telles que les capacités d'écoute, les compétences relationnelles, les capacités de négociation et de communication, la maîtrise d'au moins une langue étrangère.}
      \end{cvitems}
    }

%---------------------------------------------------------
  \cventry
    {BTS CIRA}
    {Lycée Louis Vincent} % Institution
    {Metz, France} % Location
    {09/1994 - 09/1996}
    {
      Le titulaire du Brevet de Technicien Supérieur en Contrôle Industriels et Regulations Automatiques
      possède toutes les compétences nécessaires pour résoudre les dysfonctionnements liés à l'automatisation 
      des équipements et des principaux procédés industriels de fabrication continue dont le cycle de 
      production ne peut être arrêté. Il maîtrise les phases de commande, de contrôle et de régulation.
    }

\end{cventries}
